\chapter{Templates e Heran\c{c}a}
Neste capítulo será discutido

\section{Código-fonte}
\href{https://github.com/hrausch/academico-django-web/tree/aula-04}{https://github.com/hrausch/academico-django-web/tree/aula-04}

\section{Requisitos de software a ser implementado}
Implementar um template com um menu lateral, e no centro o conteúdo a ser exibido, que são:
\begin{itemize}
    \item Um calendário com uma visão semanal ou mensal.
    \item Uma lista de disciplinas cadastradas em formas de card e um botão para cadastrar novas disciplinas.
    \item Além disso, no card de cada disciplina haverá opção de detalhamento e de exclusão da disciplina.
\end{itemize}
Tais interfaces podem ser vizualizadas nas figuras


\section{Etapas de desenvolvimento}

\subsection{Criar o módulo}
Este modulo que está sendo desenvolvido englobará as funcionalidades de planejamento de aulas do professor, e o nome dele será \texttt{planner}. Assim, para criá-lo faremos o comando.

\texttt{python manage.py startapp planner}

Importante ressaltar, conforme visto na seção \ref{sec:criarapp}, ao criar um módulo é necessário atualizar o \texttt{settings.py} adicionando o novo módulo na variável \texttt{INSTALLED\_APPS}. 
\begin{lstlisting}[language=Python]
    INSTALLED_APPS = [
    'planner',
    'landing_page',
    'django.contrib.admin',
    'django.contrib.auth',
    'django.contrib.contenttypes',
    'django.contrib.sessions',
    'django.contrib.messages',
    'django.contrib.staticfiles',
]

\end{lstlisting}

\subsection{HTML das páginas}
\begin{enumerate}
    \item Dentro do diretório do aplicativo cria-se a pasta \texttt{templates/planner}.
    \item OBS 1: Na aula anterior adicionamos o arquivo html dentro da pasta templates, contudo, para que o Django utilize a sua máquina de forma automática, precisamos criar dentro da pasta templates uma nova pasta com o nome do módulo.
\item OBS 2: Deve-se alterar, também, o módulo landing\_page.
\end{enumerate}

\subsubsection{Configurar settings.py}
Essa configuração permite que o Django encontre os arquivos de templates corretamente, mesmo quando há múltiplos apps com templates semelhantes.
\begin{lstlisting}[language=Python]
    TEMPLATES = [
    {
        'BACKEND': 'django.template.backends.django.DjangoTemplates',
        'DIRS': [BASE_DIR / 'planner/templates'],
        'APP_DIRS': True,
        'OPTIONS': {
            'context_processors': [
                'django.template.context_processors.debug',
                'django.template.context_processors.request',
                'django.contrib.auth.context_processors.auth',
                'django.contrib.messages.context_processors.messages',
            ],
        },
    },
]

\end{lstlisting}

\subsubsection{HTML}

adicionar

\subsection{Criar as views}
\begin{lstlisting}[language=Python]
def admin_dashboard(request):
   return render(request, 'planner/admin_dashboard.html')

def disciplinas(request):
   return render(request, 'planner/disciplinas.html')
\end{lstlisting}
\subsection{Criar as rotas}
\subsubsection{Rota globla do projeto}
\begin{lstlisting}[language=Python]
    urlpatterns = [
   # path('admin/', admin.site.urls),
   path("", include("landing_page.urls")),
   path("", include("planner.urls")),
]

\end{lstlisting}

\subsubsection{Rota do módulo}
\begin{lstlisting}[language=Python]
from django.urls import path
from . import views

urlpatterns = [
   path('dashboard/', views.admin_dashboard, name='admin_dashboard'),
   path('disciplinas/', views.disciplinas, name='disciplinas'),
]

    
\end{lstlisting}

\section{Templates}
\begin{itemize}
  \item Criar estrutura: \texttt{templates/app\_name/}
  \item Configurar \texttt{TEMPLATES} no \texttt{settings.py}.
  \item Criar \texttt{base.html} com layout padr\~ao.
  \item Usar \texttt{{\% extends \textquotedbl{}base.html\textquotedbl{} \%}} e \texttt{{\% block content \%}} nos templates.
\end{itemize}

\section{Exemplo de Views e Rotas}


\section{Exercício de fixação}
Crie uma interface para gestão de tarefas. A tela terá um botão de adicionar tarefas que abrirá uma modal com os campos: título, data de entrega, prioridade.
Logo abaixo terá uma tabela com a lista de tarefas exibindo os dados cadastrados e com um botão de exclusão.

